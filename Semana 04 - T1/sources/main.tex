\documentclass{article}
\usepackage[utf8]{inputenc}

\title{Revisão da Literatura}
\author{Yuri Bittencourt, Anderson Gralha \\
        Pontifícia Universidade Católica do Rio Grande do Sul 
	(PUCRS) \\ Porto Alegre -- RS -- Brazil
	 }

\date{Abril 2019}

\usepackage{hyperref}
\usepackage{natbib}
\usepackage{graphicx}
\usepackage{indentfirst}

\setlength{\parindent}{4em}
\setlength{\parskip}{1em}
\bibliographystyle{alpha}
\begin{document}
\righthyphenmin 62
\maketitle

\section{Introducão}
Este trabalho tem como objetivo realizar uma revisão da literatura sob o assunto de engenharia de software orientada à modelos no período compreendido entre 2016 e 2019, apresentando uma análise dos artigos e o motivo de escolha dos mesmos.

\section{Artigos}
\subsection{ThingML: A language and code generation framework for heterogeneous targets}

Referência escolhida pois apresenta e discute sobre o framework ThingML que é um framework gerador de código altamente customizável e multiplataforma para plataformas heterogêneas. \par
No artigo \citep{Harrand:2016:TLC:2976767.2976812} é tratado  como a abordagem do ThingML é refinada iterativamente, apresenta o seu framework de geração de código propriamente dito, disserta sobre as linguagens suportadas e seu funcionamento. Brevemente disserta sobre as versões do ThingML e sua evolução, apresenta um esquema organizacional dos módulos que compõe o mesmo. Além disso o artigo ainda mostra cases de softwares construídos com ele. \par

\textbf{Palavras-chave:} MDSE - Model Driven Software Engineering - Code Generation.\\
\href{https://www.scopus.com/record/display.uri?eid=2-s2.0-85008457888&origin=resultslist&sort=cp-f&src=s&st1=MDSE+or+22Model+Driven+Software+Engineering%22&nlo=&nlr=&nls=&sid=932f64cfcf6c13206bf91a78ce2f53c8&sot=b&sdt=cl&cluster=scopubyr%2c%222019%22%2ct%2c%222018%22%2ct%2c%222017%22%2ct%2c%222016%22%2ct&sl=58&s=TITLE-ABS-KEY%28MDSE+or+%22Model+Driven+Software+Engineering%22%29&relpos=0&citeCnt=19&searchTerm=}{Link Scopus} 
E \href{https://dl.acm.org/citation.cfm?doid=2976767.2976812}{Link Publisher}

\subsection{Software engineering in start-up companies: An analysis of 88 experience reports}
Referência escolhida pois o demonstram um panorama de empresas grandes e bem sucedidas, que teoricamente investem muito em processos de engenharia de software.\par
Neste Artigo \citep{Klotins2019} é demonstrado que nas experiências reportadas a maioria das empresas aplica elementos de engenharia de software relacionados a engenharia de requerimentos, design de software e qualidade de software, além do aspecto de negócios que é amplamente aplicado com relação a visão da empresa e estratégias de desenvolvimento. Este artigo relaciona-se com a disciplina pois relaciona diretamente boas práticas de geração de modelos com o sucesso das startups. \par

\textbf{Palavras-chave:} Model Driven Software Engineering.\\
\href{https://www.scopus.com/record/display.uri?eid=2-s2.0-85047198507&origin=resultslist&sort=plf-f&src=s&st1=%22software+engineering%22&st2=%22Validation%22+or+%22Verification%22+or+%22rastreability%22+or+%22other+applications%22&nlo=&nlr=&nls=&sid=9ffc2a7f5123c053c2fc0e6507d97681&sot=b&sdt=b&sl=122&s=%28TITLE-ABS-KEY%28%22software+engineering%22%29+AND+ALL%28%22Validation%22+or+%22Verification%22+or+%22rastreability%22+or+%22other+applications%22%29%29&relpos=57&citeCnt=2&searchTerm=}{Link Scopus} E
\href{https://link.springer.com/article/10.1007%2Fs10664-018-9620-y}{Link Publisher}

\subsection{Engineering Algorithms for Scalability through Continuous Validation of Performance Expectations}
Referência selecionada pois fala sobre como modelos e engenharia de software pode ser aplicada ao conceito de integração contínua e escalabilidade. \par
O artigo \cite{8632716} propõe uma nova abordagem para engenharia de software aplicada a sistemas de grande escala. A pesquisa identifica problemas em escalabilidade e os algoritmos aplicados, bem como os bugs e espaços para melhorias nestes algoritmos. O artigo demonstra um método que permite encontrar problemas antecipadamente do que seria encontrado utilizando outros métodos. Este artigo relaciona-se com a disciplina pois ele fala como modelos bem aplicados podem ajudar na escalabilidade do projeto, utilizando-se de frameworks que auxiliam na criação de modelos com foco na escalabilidade. \par

\textbf{Palavras-chave:} Model Driven Software Engineering.\\
\href{https://www.scopus.com/record/display.uri?eid=2-s2.0-85061028501&origin=resultslist&sort=plf-f&src=s&st1=%22software+engineering%22&st2=%22Validation%22+or+%22Verification%22+or+%22rastreability%22+or+%22other+applications%22&nlo=&nlr=&nls=&sid=c679c7514cf1dd66899b17b61d8d81ed&sot=b&sdt=b&sl=132&s=%28TITLE-ABS-KEY%28%22software+engineering%22%29+AND+TITLE-ABS-KEY%28%22Validation%22+or+%22Verification%22+or+%22rastreability%22+or+%22other+applications%22%29%29&relpos=50&citeCnt=0&searchTerm=}{Link Scopus} E 
\href{https://ieeexplore.ieee.org/document/8632716/}{Link Publisher}

\bibliography{references}
\end{document}
